% ----------------------------------------------------------------------
\begin{frame}{Motivation}
  \bigskip
  \begin{itemize}
  \item<1-> \structure{Question} \
    Is there a propositional formula $F(P)$ such that the models of~$F(P)$ correspond to the
    stable models of $P$~?
    \bigskip
  \item<2-> \structure{Observation} \
    Although each atom is defined through a set of rules,
    each such rule provides only a \alert{sufficient} condition for its head atom
    \medskip
  \item<3-> \structure{Idea} \
    The idea of program completion is to turn such implications into a definition
    by adding the corresponding \alert{necessary} counterpart
  \end{itemize}
\end{frame}
% ----------------------------------------------------------------------
\begin{frame}{Program completion}
  \bigskip
  Let $P$ be a normal logic program
  \bigskip
  \begin{itemize}
  \item
    The \alert{completion} \CF{P} of $P$ is defined as follows
    \[
    \CF{P}
    =
    \left\{a \leftrightarrow \textstyle{\bigvee_{r\in P, \head{r} = a}} \BF{\body{r}} \mid a\in\atom{P}\right\}
    \]
    where
    \[
    \BF{\body{r}}
    =
    \textstyle{\bigwedge_{a\in\pbody{r}}}      a
    \wedge
    \textstyle{\bigwedge_{a\in\nbody{r}}} \neg a
    \]
  \end{itemize}
\end{frame}
%----------------------------------------------------------------
\begin{frame}{An example}
\[
P
=
\left\{
  \begin{array}{l}
    a \leftarrow                  \\
    b \leftarrow \neg a          \\
    c \leftarrow a, \neg d       \\
    d \leftarrow \neg c, \neg e \\
    e \leftarrow b, \neg f       \\
    e \leftarrow e
  \end{array}
\right\}
\pause\qquad
\CF{P}
=
\left\{
  \begin{array}{l}
    a \leftrightarrow \top                  \\
    b \leftrightarrow \neg{a}               \\
    c \leftrightarrow a\wedge \neg{d}       \\
    d \leftrightarrow \neg{c}\wedge \neg{e} \\
    e \leftrightarrow (b\wedge \neg{f}) \vee e\\
    f \leftrightarrow \bot
  \end{array}
\right\}
\]
\end{frame}
% ----------------------------------------------------------------
\begin{frame}{A closer look}
  \bigskip
  \begin{itemize}
  \item \CF{P} is logically equivalent to $\CFIF{P}\cup\CFFI{P}$,
    where

    \begin{eqnarray*}
      \CFIF{P}
      & = &
      % \left\{a \leftarrow \textstyle{\bigvee_{r\in P, \head{r} = a}} \BF{\body{r}} \mid a\in\atom{P}\right\}
      \left\{a \leftarrow \textstyle{\bigvee_{B\in\atbody{P}{a}}} \BF{B} \mid a\in\atom{P}\right\}
      \\
      \CFFI{P}
      & = &
      % \left\{a \rightarrow \textstyle{\bigvee_{r\in P, \head{r} = a}} \BF{\body{r}} \mid a\in\atom{P}\right\}
      \left\{a \rightarrow \textstyle{\bigvee_{B\in\atbody{P}{a}}} \BF{B} \mid a\in\atom{P}\right\}
      \\[10pt]
      \atbody{P}{a}
      & = &
      \left\{\body{r} \mid r\in P\text{ and }\head{r}=a\right\}
    \end{eqnarray*}
    \medskip
  \item<2-> \CFIF{P} characterizes the classical models of $P$
  \item<2-> \CFFI{P} completes $P$ by adding necessary conditions for all atoms
\end{itemize}
\end{frame}
% ----------------------------------------------------------------
\begin{frame}{A closer look}
\[
\only<1-2>{%
P
=
 \left\{
    \begin{array}{l}
      a \leftarrow                  \\
      b \leftarrow \neg a          \\
      c \leftarrow a, \neg d       \\
      d \leftarrow \neg c, \neg e \\
      e \leftarrow b, \neg f       \\
      e \leftarrow e
    \end{array}
  \right\}
\qquad}
\only<2->{%
\CFIF{P}
=
 \left\{
    \begin{array}{l}
      a \leftarrow \top                  \\
      b \leftarrow \neg{a}               \\
      c \leftarrow a\wedge \neg{d}       \\
      d \leftarrow \neg{c}\wedge \neg{e} \\
      e \leftarrow (b\wedge \neg{f}) \vee e\\
      f \leftarrow \bot
    \end{array}
  \right\}
}
\only<4->{%
 \left\{
    \begin{array}{l}
      a \rightarrow \top                  \\
      b \rightarrow \neg{a}               \\
      c \rightarrow a\wedge \neg{d}       \\
      d \rightarrow \neg{c}\wedge \neg{e} \\
      e \rightarrow (b\wedge \neg{f}) \vee e\\
      f \rightarrow \bot
    \end{array}
  \right\}
  =
  \CFFI{P}}
\]
\[
\only<5->{%
  \CF{P}
  =
  \left\{
    \begin{array}{l}
      a \leftrightarrow \top                  \\
      b \leftrightarrow \neg{a}               \\
      c \leftrightarrow a\wedge \neg{d}       \\
      d \leftrightarrow \neg{c}\wedge \neg{e} \\
      e \leftrightarrow (b\wedge \neg{f}) \vee e\\
      f \leftrightarrow \bot
    \end{array}
  \right\}}
\only<5>{\phantom{ = \CF{P} }}
\only<6>{\quad\equiv\quad\CFIF{P}\cup\CFFI{P}}
\]
\end{frame}
%----------------------------------------------------------------
\begin{frame}{Supported models}
  \bigskip
  \begin{itemize}
  \item<1-> Every stable model of $P$ is a model of \CF{P}\pause[2]
    but not vice versa
    \smallskip
  \item<3-> Models of \CF{P} are called the \alert{supported models} of $P$
    \bigskip
  \item<4-> In other words, every stable model of $P$ is a supported model of $P$
    \smallskip
  \item<4-> By definition, every supported model of $P$ is also a model of $P$
\end{itemize}
\end{frame}
% ----------------------------------------------------------------
\begin{frame}{An example}
  \bigskip
  \begin{itemize}
  \item<1-> []
    \par\bigskip
    \(
    P
    =
    \left\{
      \begin{array}{lll}
        a \leftarrow                \quad &
        c \leftarrow a, \neg d      \quad &
        e \leftarrow b, \neg f
        \\
        b \leftarrow \neg a         \quad &
        d \leftarrow \neg c, \neg e \quad &
        e \leftarrow e
      \end{array}
    \right\}
    \)
    \bigskip
    \bigskip
  \item<2-> $P$ has 21           models, including $\{a,c\}$, $\{a,d\}$, but also $\{a,b,c,d,e,f\}$
    \smallskip
  \item<3-> $P$ has  3 supported models, namely $\{a,c\}$, $\{a,d\}$, and $\{a,c,e\}$
    \smallskip
  \item<4-> $P$ has  2 stable    models, namely $\{a,c\}$ and $\{a,d\}$
  \end{itemize}
\end{frame}
% ----------------------------------------------------------------
%
%%% Local Variables:
%%% mode: latex
%%% TeX-master: "../../main"
%%% End:
