% ----------------------------------------------------------------------
\begin{frame}{The mismatch}
  \begin{itemize}
  \item<1-> \structure{Question} \
    What causes the mismatch between supported\\ and stable models?
      \smallskip
  \item<2->[] \itarrow\
    Consider the unstable yet supported model $\{a,c,e\}$ above~!
    \medskip
  \item<3-> \structure{Answer} \
    The mismatch between supported and stable models\\ is caused by cyclic derivations
      \smallskip
    \begin{itemize}\normalsize
    \item<4-> Atoms in a stable model can    be ``derived'' from a program\\ in a finite number of steps
      \smallskip
    \item<5-> Atoms in a        cycle cannot be ``derived'' from a program\\ in a finite number of steps
      \only<6>{\footnote{unless being ``supported from outside the cycle''}}
      \medskip
    \item<8-> \structure{Note} \
      Such atoms do not contradict the completion of a program and
      do thus not eliminate an unstable supported model
    \end{itemize}
  \end{itemize}
\end{frame}
% ----------------------------------------------------------------------
\begin{frame}{Non-cyclic derivations}
  \medskip
  Let $X$ be a stable model of normal logic program $P$
  \medskip
  \begin{itemize}
  \item<2-> For every atom $a\in X$,
    there is a finite sequence of positive rules
    \[
    \langle r_1,\dots,r_n\rangle
    \]
    such that
    \begin{enumerate}\normalsize
    \item $r_i\in\reduct{P}{X}$                                 \hfill for $1\leq i\leq n\qquad$
    \item $\pbody{r_i}\subseteq\{\head{r_j} \mid 0\leq j< i\}$  \hfill for $1\leq i\leq n\qquad$
    \item $\head{r_n}=a$
    \end{enumerate}
    \smallskip
  \item<3-> That is,
    each atom of $X$ has a non-cyclic derivation from \reduct{P}{X}
    \medskip
  \item<4-> \structure{Example} \
    There is no finite sequence of rules providing a derivation\\ for $e$ from $P^{\{a,c,e\}}$ as given above
  \end{itemize}
\end{frame}
% ----------------------------------------------------------------------
\begin{frame}{Positive atom dependency graph}
  \bigskip
  \begin{itemize}
  \item<2->
    The origin of (potential) circular derivations can be read off the
    \alert{positive atom dependency graph} $G(P)$ of a logic program $P$
    \[
      (\atom{P}, \{(b,a) \mid r\in P, b \in \pbody{r}, \head{r}=a\})
    \]
    \smallskip
  \item<3-> A logic program $P$ is called \alert{tight}, if $G(P)$ is acyclic
  \end{itemize}
\end{frame}
% ----------------------------------------------------------------------
\begin{frame}{Example}
  \bigskip
  \begin{itemize}
  \item<1-> []
    \par
    \(
    P
    =
    \left\{
      \begin{array}{lll}
        a \leftarrow                 \quad &
        c \leftarrow a, \neg d       \quad &
        e \leftarrow b, \neg f
        \\
        b \leftarrow \neg a         \quad &
        d \leftarrow \neg c, \neg e \quad &
        e \leftarrow e
      \end{array}
    \right\}
    \)
    \bigskip
  \item<2->
    \(
    G(P)= (\{a,b,c,d,e,f\},\{(a,c),(b,e),(e,e)\})
    \)
  \item<3-> []
    \begin{center}
        \begin{tikzpicture}[
    >=stealth',
    node/.style={draw,circle,minimum size=16pt,inner sep=0},
    -> ]
    \matrix (m) [matrix of math nodes, row sep=1em, column sep=1em, nodes={node}] {
      a & c & d \\
      b & e & f \\ };
    \path 
    (m-1-1) edge (m-1-2)
    (m-2-1) edge (m-2-2)
    (m-2-2) edge [loop below] ();
  \end{tikzpicture}
%%% Local Variables: 
%%% mode: latex
%%% TeX-master: "../asp"
%%% End: 

    \end{center}
    \smallskip
  \item<4-> $P$ has supported models: $\{a,c\}$, $\{a,d\}$, and $\{a,c,e\}$
    \smallskip
  \item<4-> $P$ has stable    models: $\{a,c\}$ and $\{a,d\}$
  \end{itemize}
\end{frame}
% ----------------------------------------------------------------------
\begin{frame}{Tight logic programs}
  \bigskip
  \begin{itemize}
  \item<1-> A logic program $P$ is called \alert{tight}, if $G(P)$ is acyclic
    \medskip
  \item<2-> For tight programs, stable and supported models coincide\pause[3]
    \bigskip
  \item<3->[] \par
    \begin{minipage}[t]{0.8\linewidth}
    \begin{block}{Fages' Theorem}
      \smallskip
      Let $P$ be a tight normal logic program and $X\subseteq\atom{P}$
      % \par\medskip
      \\
      Then, $X$ is a stable model of~$P$ iff $X\models\CF{P}$
    \end{block}
    \end{minipage}
  \end{itemize}
\end{frame}
% ----------------------------------------------------------------------
\begin{frame}{Another example}
  \bigskip
  \begin{itemize}
  \item<1-> []
    \par\bigskip
    \(
    P
    =
    \left\{
      \begin{array}{llll}
        a \leftarrow \neg b
      & c \leftarrow a,b
      & d \leftarrow a
      & e \leftarrow\neg a,\neg b
      \\
        b \leftarrow \neg a
      & c \leftarrow d
      & d \leftarrow b,c
      &
      \end{array}
    \right\}
    \)
    \bigskip
    \bigskip
  \item<2->
    \(
    G(P)= (\{a,b,c,d,e\},\{(a,c),(a,d),(b,c),(b,d),(c,d),(d,c)\})
    \)
  \item<3-> []
    \begin{center}
      % ----------------------------------------------------------------------
\begin{tikzpicture}[
  >=stealth',
  node/.style={draw,circle,minimum size=16pt,inner sep=0},
  -> ]
  \matrix (m) [matrix of math nodes, row sep=1em, column sep=1em, nodes={node}] {
    d & a & c & e \\
      & b &   &   \\ };
  \path
  (m-1-1) edge[<->, bend left=45] (m-1-3)
  (m-1-2) edge (m-1-1)
  (m-1-2) edge (m-1-3)
  (m-2-2) edge (m-1-1)
  (m-2-2) edge (m-1-3);
\end{tikzpicture}
% ----------------------------------------------------------------------
%
%%% Local Variables:
%%% mode: latex
%%% TeX-master: "../../../main"
%%% End:

    \end{center}
  \item<4-> $P$ has supported models: $\{a,c,d\}$, $\{b\}$, and $\{b,c,d\}$
    \smallskip
  \item<4-> $P$ has stable    models: $\{a,c,d\}$ and $\{b\}$
\end{itemize}
\end{frame}
% ----------------------------------------------------------------------
%
%%% Local Variables:
%%% mode: latex
%%% TeX-master: "../../main"
%%% End:
